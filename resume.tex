%%%%%%%%%%%%%%%%%
% This is an example CV created using altacv.cls (v1.1, 21 November 2016) written by
% LianTze Lim (liantze@gmail.com), based on the 
% Cv created by BusinessInsider at http://www.businessinsider.my/a-sample-resume-for-marissa-mayer-2016-7/?r=US&IR=T
% 
%% It may be distributed and/or modified under the
%% conditions of the LaTeX Project Public License, either version 1.3
%% of this license or (at your option) any later version.
%% The latest version of this license is in
%%    http://www.latex-project.org/lppl.txt
%% and version 1.3 or later is part of all distributions of LaTeX
%% version 2003/12/01 or later.
%%%%%%%%%%%%%%%%

%% If you want to use \orcid or the
%% academicons icons, add "academicons"
%% to the \documentclass options. 
%% Then compile with XeLaTeX or LuaLaTeX.
% \documentclass[10pt,a4paper,academicons]{altacv}
\documentclass[10pt,a4paper]{altacv}

%% AltaCV uses the fontawesome and academicon fonts
%% and packages. 
%% See texdoc.net/pkg/fontawecome and http://texdoc.net/pkg/academicons for full list of symbols.
%% When using the "academicons" option,
%% Compile with LuaLaTeX for best results. If you
%% want to use XeLaTeX, you may need to install
%% Academicons.ttf in your operating system's font %% folder.


% Change the page layout if you need to
\geometry{left=1cm,right=9cm,marginparwidth=6.8cm,marginparsep=1.2cm,top=1cm,bottom=1cm}

% Change the font if you want to.

% If using pdflatex:
\usepackage[utf8]{inputenc}
\usepackage[T1]{fontenc}
\usepackage[default]{lato}
\RequirePackage[hidelinks]{hyperref}

% If using xelatex or lualatex:
% \setmainfont{Lato}

% Change the colours if you want to
\definecolor{VividPurple}{HTML}{3E0097}
\definecolor{SlateGrey}{HTML}{2E2E2E}
\definecolor{LightGrey}{HTML}{666666}

\definecolor{Blue}{HTML}{4e7f99}
\definecolor{Red}{HTML}{c2524f}
\definecolor{Green}{HTML}{577843}

\colorlet{heading}{Blue}
\colorlet{accent}{Red}
\colorlet{accent2}{Blue}
\colorlet{emphasis}{SlateGrey}
\colorlet{body}{LightGrey}

% Change the bullets for itemize and rating marker
% for \cvskill if you want to
\renewcommand{\itemmarker}{{\small\textbullet}}
\renewcommand{\ratingmarker}{\faCircle}


%% sample.bib contains your publications
\addbibresource{sample.bib}

\begin{document}
\name{Marcelo Prates, PhD}
\tagline{Data Scientist \& Machine Learning Engineer}
\personalinfo{
  \email{\href{mailto:marceloorp@gmail.com}{marceloorp@gmail.com}}
  \linkedin{\href{https://linkedin.com/in/marceloprates}{marceloprates}}
  \github{\href{https://github.com/marceloprates}{marceloprates}}
}

%% Make the header extend all the way to the right, if you want. Extend the right margin by 8cm (=6.8cm marginparwidth + 1.2cm marginparsep)
\begin{adjustwidth}{}{-8cm}
\makecvheader
\end{adjustwidth}

%% Provide the file name containing the sidebar contents as an optional parameter to \cvsection.
%% You can always just use \marginpar{...} if you do
%% not need to align the top of the contents to any
%% \cvsection title in the "main" bar.
\cvsection[page1sidebar]{Experience (6 years)}

{\small
%\vspace{25}

%\cvevent{ML Engineer \& Project lead}{IDS Software}{Jun 2023 -- Now}{Pato Branco, Paran\'a, Brazil (Remote)}
%\begin{itemize}c
%\item Development of a Large Language Model (LLM) solution for automating customer service communication.
%\end{itemize}

\cvevent{Lead Data Scientist}{Dataside}{Oct 2023 -- Now}{São José dos Campos, Brazil (Remote)}
\begin{itemize}
\item Designed and implemented a large-scale enterprise LLM provider for a Forbes Global 100 company.
\item Developed ML/CV/NLP solutions for photogrammetry, computer vision, advanced OCR techniques on highly unstructured data, causal time series forecasting, classification, regression and clustering, including unsupervised and semi-supervised approaches.
\item Built GenAI solutions RAG, CAG, Text-to-SQL, and multi-agent LLM systems.
\item Mentored 5 junior data scientists.
\item Contributed to sales efforts by showcasing data science capabilities to potential clients.
\item Tech stack: Pytorch, Pytorch Lightning, Pandas, Scikit-learn, Numpy, Jupyter Notebooks, Azure OpenAI, Azure Databricks, HuggingFace Transformers, LangChain, LangGraph, BertTopic, CrewAI.
\end{itemize}

\divider

\cvevent{AI/LLM Consultant}{Vortigo}{Nov 2023 -- Jun 2024}{Porto Alegre, Brazil (Remote)}
\begin{itemize}
\item Designed and implemented advanced assistant ChatBots leveraging OpenAI's GPT-3.5 and GPT-4, enhancing customer service automation with proprietary source code and spreadsheet knowledge bases.
\end{itemize}

\divider

\cvevent{Sabbatical for Generative Art \& Open Source}{Apr 2023 -- Oct 2023}{Porto Alegre, Brazil}{}
\begin{itemize}
\item Advanced generative art projects, creating innovative pieces.
\item Enhanced and maintained the open-source Python package \href{https://github.com/marceloprates/prettymaps}{prettymaps}.
\item Refined Python and creative coding skills, mixing my data science expertise with my artistic process.
\end{itemize}

\divider

\cvevent{Computer Vision Consultant}{ConstructIN}{Mar 2022 -- Apr 2023}{Porto Alegre, Brazil (Remote)}
\begin{itemize}
\item Developed advanced CV solutions for automated construction site monitoring using 360-degree photography.
\item Led a team of 3 data scientists, creating 4 CV applications, including a client dashboard with a semantic segmentation module.
\end{itemize}

\divider

\cvevent{Generative Art Instructor}{Responsive Cities Workshop}{Nov 2022}{Porto Alegre, Brazil}
\begin{itemize}
\item Taught a course on generative art, covering its history, key concepts, and practical training with popular tools and libraries.
\end{itemize}

\divider

\cvevent{Principal Data Scientist}{Condati}{Nov 2021 -- Dec 2023}{Menlo Park, California (Remote)}
\begin{itemize}
\item Enhanced ML solutions for optimizing digital marketing bids, significantly improving financial returns.
\item Resolved model performance issues, boosting efficiency and effectiveness.
\item Collaborated with cross-functional teams for seamless integration and deployment.
\item Tech stack: MySQL, Julia, Python, Tensorflow.jl, Torch.jl, MLJ.jl, Flux.jl, Scipy, Numpy, Pandas, Pytorch, Jupyter Notebooks, AWS Sagemaker.
\end{itemize}

\divider

\cvevent{AI Researcher \& Project Lead - ML for Health}{Samsung Research Brazil}{Mar 2020 -- Nov 2021}{Campinas, Brazil (Remote)}
\begin{itemize}
\item Developed an ML-powered solution for real-time health monitoring on Samsung Galaxy Watch 4.
\item Designed data collection protocols and researched AI techniques for health monitoring.
\item Prototyped and tested ML solutions for high accuracy and reliability.
\item Implemented a memory-constrained solution in C for deployment.
\item Led a multidisciplinary team, integrating the app into the Galaxy Watch line globally.
\item Tech stack: Python, Julia, C/C++, TensorFlow \& Keras, PyTorch \& PyTorch Lightning, Pandas, SciPy, NumPy, Scikit-learn, Matplotlib, MLJ.jl, Flux.jl, Jupyter Notebooks, AWS Sagemaker, C, ONNX, custom Python-to-C transpilers, Samsung's Tizen OS
\end{itemize}

\divider

\cvevent{Data Scientist}{Poatek IT Consulting}{Jun 2019 -- Mar 2020}{Porto Alegre, Brazil}
\begin{itemize}
\item Solved combinatorial optimization problems in vehicle routing.
\item Applied computer vision and NLP for document cataloging and data analysis.
\item Conducted NER and sentiment analysis for actionable insights.
\item Developed credit risk models for financial predictions.
\item Analyzed geospatial data for decision-making.
\end{itemize}

\divider

%\cvevent{Volunteer Mathematics Teacher}{ONGEP Pre-Vestibular Popular (NGO)}{Apr 2018 -- Apr 2020}{Porto Alegre, Brazil}
%\begin{itemize}
%\item Maths teaching for a Brazilian preparatory university exam
%\end{itemize}

\cvevent{PhD in Computer Science}{Federal University of Rio Grande do Sul (UFRGS)}{Aug 2015 -- Jul 2019}{Porto Alegre, Brazil}
\begin{itemize}
\item \textbf{AI Ethics}: {\small {\href{http://www.easychair.org/publications/paper/Z7D4}{On Quantifying and Understanding the Role of Ethics in AI Research (GCAI 2018)}}}, {\small {\href{http://arxiv.org/abs/1809.02208}{Assessing Gender Bias in Machine Translation (NCA 2020)}}}
    \item \textbf{Graph Deep Learning \& Combinatorial Optimization}: {\small {\href{http://arxiv.org/abs/1809.02721}{Learning to Solve NP-Complete Problems (AAAI 2019)}}}, {\small {\href{https://arxiv.org/pdf/1903.04598.pdf}{Graph neural networks meet neural-symbolic computing (IJCAI 2021)}}}
\end{itemize}
\divider
}
%\cvevent{BSc in Computer Science}{Federal University of Rio Grande do Sul (UFRGS)}{2012 -- 2015}{Porto Alegre, Brazil}
%\begin{itemize}
%\item \textbf{Social \& Human Computation}: {\small \href{http://www.aaai.org/ocs/index.php/AAAI/AAAI15/paper/view/9844}{Collaboration in Social Problem-Solving (AAAI 2015)}}, {\small \href{http://www.aaai.org/ocs/index.php/HCOMP/HCOMP13/paper/view/7481}{Leveraging Collaboration (HCOMP 2013)}}
%\end{itemize}

%\vspace{35}
%\begin{figure}[h!]
%    \centering
%    \includegraphics[width=.85\linewidth]{wobble-2-mosaic-v2-CV-2}
%\end{figure}

Figures created by me. Resume was created using \href{https://github.com/liantze/AltaCV}{AltaCV} class by LianTze Lim (liantze@gmail.com).

%\cvsection{A Work Day of My Life}

% Adapted from @Jake's answer from http://tex.stackexchange.com/a/82729/226
% \wheelchart{outer radius}{inner radius}{
% comma-separated list of value/text width/color/detail}
%\wheelchart{1.5cm}{0.5cm}{%
%  10/13em/accent!30/Business Planning and Strategy implementation, 
%  25/9em/accent!60/Resolving issues with project teams \& stakeholders,
%  5/12em/accent!10/Budget Management, 
%  20/12em/accent!40/Stakeholder Management,
%  5/8em/accent!20/In-House Business development,
%  30/9em/accent/Leading and motivating project members that their work has meaning,
%  5/8em/accent!20/Networking
%}

\clearpage


\end{document}
